\documentclass[12pt, reqno]{amsart}

\textwidth = 7.0 in
\textheight = 9.0 in
\oddsidemargin = -0.2 in
\evensidemargin = -0.2 in
\topmargin = -0.3 in
\headheight = 0.0 in
\headsep = 0.3 in
\parskip = 0.05 in
\parindent = 0.3 in

\usepackage{enumerate}
\usepackage{amsmath}
\usepackage{mathrsfs}
\usepackage{color}
\usepackage{bm}
\usepackage{multirow}
%\usepackage{bbm}
\def\cc{\color{blue}}
\usepackage[normalem]{ulem}

\def\avint{\mathop{\,\rlap{--}\!\!\int}\nolimits}

\title{COP 5536 - Project Report}
\author[James Diffenderfer]{James Diffenderfer \\ jdiffen1@ufl.edu \\ UFID: 6191-4098}
\date{\today}

\begin{document}
\maketitle

\section*{Introduction}

Here we include all function prototypes for this implementation of the Fibonacci heap. The two classes defined for this implementation of the Fibonacci heap are a \emph{Node} class and the \emph{Fibonacci Heap} class.

\section*{Node Class}
In this section, we provide the contents and functions used for the \emph{Node} class which are used as nodes for the Fibonacci heap implementation. We first provide the contents of each node followed by the function prototypes in the \emph{Node} class. 

\noindent {\bfseries Class Contents:}
\begin{center}
    \begin{tabular}{| l | l | p{11cm} |}
    \hline
    {\bfseries Parameters} & {\bfseries Type} & {\bfseries Description} \\ \hline
    data & \texttt{long int} & Data in node (keyword frequency) \\ \hline
    degree & \texttt{long int} & Degree of node (number of children) \\ \hline
    keyword & \texttt{string} & Keyword at node \\ \hline
    childcut & \texttt{bool} & Indicates whether node has lost child since becoming child of current parent \\ \hline
    child & \texttt{Node *} & Child pointer \\ \hline
    lsibling & \texttt{Node *} & Left sibling pointer \\ \hline
    rsibling & \texttt{Node *} & Right sibling pointer \\ \hline
    parent & \texttt{Node *} & Parent pointer \\ \hline
    \end{tabular}
\end{center}
\vspace{5mm}

\noindent {\bfseries Function Prototypes:}

{\large \texttt{void merge(Node *nd)}}
\begin{center}
    \begin{tabular}{| l | p{2.5cm} | p{9.5cm} |}
    \hline
    {\bfseries Description} & \multicolumn{2}{p{12.5cm}|}{Void function which makes tree rooted at given node a subtree of the current node.} \\ \hline
    \multirow{1}{*}{\bfseries Parameters} & nd & Pointer to root of tree which is to be made subtree of current node \\ \hline
    {\bfseries Return value} & \multicolumn{2}{p{12.5cm}|}{No values are returned} \\ \hline
    \end{tabular}
\end{center}

\newpage
\section*{Fibonacci Heap Class}
In this section, we provide the contents and functions used for the \emph{Fibonacci Heap} class. We first provide the contents of each Fibonacci heap followed by the function prototypes in the \emph{Fibonacci Heap} class.

\noindent {\bfseries Class Contents:}
\begin{center}
    \begin{tabular}{| l | l | p{6.8cm} |}
    \hline
    {\bfseries Parameters} & {\bfseries Type} & {\bfseries Description} \\ \hline
    n & \texttt{long int} & Number of nodes in fibonacci heap \\ \hline
    max & \texttt{Node *} & Pointer to node containing the max key \\ \hline
    hashmap & \texttt{unordered\_map<string, Node*>} & Hash table for keywords mapped to node pointer \\ \hline
    \end{tabular}
\end{center}
\vspace{5mm}

\noindent {\bfseries Function Prototypes:}

{\large \texttt{bool only\_child(Node *nd)}}
\begin{center}
    \begin{tabular}{| l | p{2.5cm} | p{9.5cm} |}
    \hline
    {\bfseries Description} & \multicolumn{2}{p{12.5cm}|}{Boolean function which checks if node has any siblings.} \\ \hline
    \multirow{1}{*}{\bfseries Parameters} & nd & Pointer to node to check if it has siblings. \\ \hline
    {\bfseries Return value} & \multicolumn{2}{p{12.5cm}|}{Returns TRUE if given node is only child FALSE if given node has at least one sibling.} \\ \hline
    \end{tabular}
\end{center} 
\vspace{3mm}   

{\large \texttt{bool barren(Node *nd)}}
\begin{center}
    \begin{tabular}{| l | p{2.5cm} | p{9.5cm} |}
    \hline
    {\bfseries Description} & \multicolumn{2}{p{12.5cm}|}{Boolean function which checks if node has any children.} \\ \hline
    \multirow{1}{*}{\bfseries Parameters} & nd & Pointer to node to check if it has children. \\ \hline
    {\bfseries Return value} & \multicolumn{2}{p{12.5cm}|}{Returns TRUE if given node has no children and FALSE if given node has at least one child.} \\ \hline
    \end{tabular}
\end{center}    
\vspace{3mm}   

{\large \texttt{Node *orphan(Node *nd)}}
\begin{center}
    \begin{tabular}{| l | p{2.5cm} | p{9.5cm} |}
    \hline
    {\bfseries Description} & \multicolumn{2}{p{12.5cm}|}{Function which checks if node has a parent.} \\ \hline
    \multirow{1}{*}{\bfseries Parameters} & nd & Pointer to node to check if it has a parent. \\ \hline
    {\bfseries Return value} & \multicolumn{2}{p{12.5cm}|}{If the given node has no parent, it returns NULL. If given node has a parent, it returns the pointer \texttt{nd->parent->child}.} \\ \hline
    \end{tabular}
\end{center}    
\vspace{3mm}   

{\large \texttt{long int rank\_bound()}}
\begin{center}
    \begin{tabular}{| l | p{2.5cm} | p{9.5cm} |}
    \hline
    {\bfseries Description} & \multicolumn{2}{p{12.5cm}|}{Compute bound on rank based on current number of nodes.} \\ \hline
    \multirow{1}{*}{\bfseries Parameters} & -- & No inputs. \\ \hline
    {\bfseries Return value} & \multicolumn{2}{p{12.5cm}|}{Returns bound on rank for rank tree.} \\ \hline
    \end{tabular}
\end{center}    
\vspace{3mm}   

{\large \texttt{void meld(Node *nd)}}
\begin{center}
    \begin{tabular}{| l | p{2.5cm} | p{9.5cm} |}
    \hline
    {\bfseries Description} & \multicolumn{2}{p{12.5cm}|}{Meld single node at root level.} \\ \hline
    \multirow{1}{*}{\bfseries Parameters} & nd & Pointer to node to be melded to root of Fibonacci heap. \\ \hline
    {\bfseries Return value} & \multicolumn{2}{p{12.5cm}|}{No value is returned.} \\ \hline
    \end{tabular}
\end{center}  
\vspace{3mm}   

{\large \texttt{void PrintRootLtoR()}}
\begin{center}
    \begin{tabular}{| l | p{2.5cm} | p{9.5cm} |}
    \hline
    {\bfseries Description} & \multicolumn{2}{p{12.5cm}|}{Print root level nodes of Fibonacci heap starting with max and moving to right.} \\ \hline
    \multirow{1}{*}{\bfseries Parameters} & -- & No inputs. \\ \hline
    {\bfseries Return value} & \multicolumn{2}{p{12.5cm}|}{No value is returned.} \\ \hline
    \end{tabular}
\end{center}
\vspace{3mm}   

{\large \texttt{long int number\_of\_nodes()}}
\begin{center}
    \begin{tabular}{| l | p{2.5cm} | p{9.5cm} |}
    \hline
    {\bfseries Description} & \multicolumn{2}{p{12.5cm}|}{Return number of nodes in Fibonacci heap.} \\ \hline
    \multirow{1}{*}{\bfseries Parameters} & -- & No inputs. \\ \hline
    {\bfseries Return value} & \multicolumn{2}{p{12.5cm}|}{Number of nodes in Fibonacci heap is returned.} \\ \hline
    \end{tabular}
\end{center}
\vspace{3mm}   

{\large \texttt{Node *find\_max()}}
\begin{center}
    \begin{tabular}{| l | p{2.5cm} | p{9.5cm} |}
    \hline
    {\bfseries Description} & \multicolumn{2}{p{12.5cm}|}{Find and return pointer to node containing maximum value in heap.} \\ \hline
    \multirow{1}{*}{\bfseries Parameters} & -- & No inputs. \\ \hline
    {\bfseries Return value} & \multicolumn{2}{p{12.5cm}|}{Returns pointer to node containing maximum value in Fibonacci heap.} \\ \hline
    \end{tabular}
\end{center}
\vspace{3mm}   

{\large \texttt{void insert(string keyword, long int frequency)}}
\begin{center}
    \begin{tabular}{| l | p{2.5cm} | p{9.5cm} |}
    \hline
    {\bfseries Description} & \multicolumn{2}{p{12.5cm}|}{Insert node into heap. If node containing keyword is already in heap, the frequency in the Fibonacci heap is increased by the input frequency amount.} \\ \hline
    \multirow{2}{*}{\bfseries Parameters} & keyword & Keyword to store in heap. \\ \cline{2-3}
    & frequency & Frequency of keyword \\ \hline
    {\bfseries Return value} & \multicolumn{2}{p{12.5cm}|}{No value is returned.} \\ \hline
    \end{tabular}
\end{center}
\vspace{3mm}   
    
{\large \texttt{void remove\_max()}}
\begin{center}
    \begin{tabular}{| l | p{2.5cm} | p{9.5cm} |}
    \hline
    {\bfseries Description} & \multicolumn{2}{p{12.5cm}|}{Remove max node from heap.} \\ \hline
    \multirow{1}{*}{\bfseries Parameters} & -- & No inputs. \\ \hline
    {\bfseries Return value} & \multicolumn{2}{p{12.5cm}|}{No value is returned.} \\ \hline
    \end{tabular}
\end{center}
\vspace{3mm}   

{\large \texttt{bool remove(Node *nd)}}
\begin{center}
    \begin{tabular}{| l | p{2.5cm} | p{9.5cm} |}
    \hline
    {\bfseries Description} & \multicolumn{2}{p{12.5cm}|}{Remove node from heap given pointer to node. Updates childcut of input parent node (if necessary).} \\ \hline
    \multirow{1}{*}{\bfseries Parameters} & nd & Pointer to node to be removed from Fibonacci heap. \\ \hline
    {\bfseries Return value} & \multicolumn{2}{p{12.5cm}|}{Returns FALSE if childcut of parent was not changed from TRUE and returns TRUE if childcut of parent node was TRUE.} \\ \hline
    \end{tabular}
\end{center}
\vspace{3mm}   
    
{\large \texttt{void increase\_key(string keyword, long int amount)}}
\begin{center}
    \begin{tabular}{| l | p{2.5cm} | p{9.5cm} |}
    \hline
    {\bfseries Description} & \multicolumn{2}{p{12.5cm}|}{Given a node, increase its frequency value by a given amount.} \\ \hline
    \multirow{2}{*}{\bfseries Parameters} & keyword & Keyword for which frequency is increased. \\ \cline{2-3}
    & frequency & Amount to increase frequency of keyword. \\ \hline
    {\bfseries Return value} & \multicolumn{2}{p{12.5cm}|}{No value is returned.} \\ \hline
    \end{tabular}
\end{center}

\end{document}

%% For function prototypes
\texttt{void merge(Node *nd)}
\begin{center}
    \begin{tabular}{| l | p{2.5cm} | p{9.5cm} |}
    \hline
    {\bfseries Description} & \multicolumn{2}{p{12.5cm}|}{Void function which makes tree rooted at given node a subtree of the current node.} \\ \hline
    \multirow{1}{*}{\bfseries Parameters} & nd & Pointer to root of tree which is to be made subtree of current node \\ \hline
    {\bfseries Return value} & \multicolumn{2}{p{12.5cm}|}{No values are returned} \\ \hline
    \end{tabular}
\end{center}



\begin{displaymath}
\phi(s) = \left\{
\begin{array}{lr}
\phi_1 (s) = (\alpha_j + s)^2 + (\beta_j + s)^2 & : s \in (- \infty, h_j - x_j - \beta_j)\\
\phi_2 (s) = (\alpha_j + s)^2 + (h_j - x_j)^2 & : s \in [h_j - x_j - \beta_j, x_j - \ell_j - \alpha_j]\\
\phi_3 (s) = (x_j - \ell_j)^2 - (h_j - x_j)^2 & : s \in (\ell_j - x_j - \alpha_j, \infty)
\end{array}
\right.
\end{displaymath}
\grid
