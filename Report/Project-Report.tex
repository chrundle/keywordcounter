\documentclass[12pt, reqno]{amsart}

\textwidth = 7.0 in
\textheight = 9.0 in
\oddsidemargin = -0.2 in
\evensidemargin = -0.2 in
\topmargin = -0.3 in
\headheight = 0.0 in
\headsep = 0.3 in
\parskip = 0.05 in
\parindent = 0.3 in

\usepackage{enumerate}
\usepackage{amsmath}
\usepackage{mathrsfs}
\usepackage{color}
\usepackage{bm}
\usepackage{multirow}
%\usepackage{bbm}
\def\cc{\color{blue}}
\usepackage[normalem]{ulem}

\def\avint{\mathop{\,\rlap{--}\!\!\int}\nolimits}

\title{COP 5536 - Project Report}
\author[James Diffenderfer]{James Diffenderfer \\ jdiffen1@ufl.edu \\ UFID: 6191-4098}
\date{\today}

\begin{document}
\maketitle

\section*{Introduction}

Here we include all function prototypes for this implementation of the Fibonacci heap.

\section*{Node Class}
In this section, we provide the contents and functions used for the \emph{Node} class. We first provide the contents of each node followed by the function prototypes in the \emph{Node} class. 

\noindent {\bfseries Class Contents:}
\begin{center}
    \begin{tabular}{| l | l | p{11cm} |}
    \hline
    {\bfseries Parameters} & {\bfseries Type} & {\bfseries Description} \\ \hline
    data & \texttt{long int} & Data in node (keyword frequency) \\ \hline
    degree & \texttt{long int} & Degree of node (number of children) \\ \hline
    keyword & \texttt{string} & Keyword at node \\ \hline
    childcut & \texttt{bool} & Indicates whether node has lost child since becoming child of current parent \\ \hline
    child & \texttt{Node *} & Child pointer \\ \hline
    lsibling & \texttt{Node *} & Left sibling pointer \\ \hline
    rsibling & \texttt{Node *} & Right sibling pointer \\ \hline
    parent & \texttt{Node *} & Parent pointer \\ \hline
    \end{tabular}
\end{center}
\vspace{2mm}

\noindent {\bfseries Function Prototypes:}

{\large \texttt{void merge(Node *nd)}}
\begin{center}
    \begin{tabular}{| l | p{2.5cm} | p{9.5cm} |}
    \hline
    {\bfseries Description} & \multicolumn{2}{p{12.5cm}|}{Void function which makes tree rooted at given node a subtree of the current node.} \\ \hline
    \multirow{1}{*}{\bfseries Parameters} & nd & Pointer to root of tree which is to be made subtree of current node \\ \hline
    {\bfseries Return value} & \multicolumn{2}{p{12.5cm}|}{No values are returned} \\ \hline
    \end{tabular}
\end{center}

\newpage
\section*{Fibonacci Heap Class}

\noindent {\bfseries Class Contents:}
\begin{center}
    \begin{tabular}{| l | l | p{6.8cm} |}
    \hline
    {\bfseries Parameters} & {\bfseries Type} & {\bfseries Description} \\ \hline
    n & \texttt{long int} & Number of nodes in fibonacci heap \\ \hline
    max & \texttt{Node *} & Pointer to node containing the max key \\ \hline
    hashmap & \texttt{unordered\_map<string, Node*>} & Hash table for keywords mapped to node pointer \\ \hline
    \end{tabular}
\end{center}
\vspace{2mm}

\noindent {\bfseries Function Prototypes:}

\end{document}

%% For function prototypes
\texttt{void merge(Node *nd)}
\begin{center}
    \begin{tabular}{| l | p{2.5cm} | p{9.5cm} |}
    \hline
    {\bfseries Description} & \multicolumn{2}{p{12.5cm}|}{Void function which makes tree rooted at given node a subtree of the current node.} \\ \hline
    \multirow{1}{*}{\bfseries Parameters} & nd & Pointer to root of tree which is to be made subtree of current node \\ \hline
    {\bfseries Return value} & \multicolumn{2}{p{12.5cm}|}{No values are returned} \\ \hline
    \end{tabular}
\end{center}





%% Scalar form of Epanechnikov kernel
Here, $\bm{y}_1, \ldots, \bm{y}_n$ denote the centers of our clusters and we start with $\bm{y}_j^{0} = \bm{x}_j$, for $1 \leq j \leq n$. Using the Epanechnikov kernel $K_h (\bm{z}) = c \left[ 1 - \frac{\| \bm{z} \|^2}{2 h^2} \right]_+$, we have that
\begin{displaymath}
K_h'(\bm{z}) = \left\{
\begin{array}{cr}
\displaystyle - \frac{c}{h^2} \bm{z}^T & : \| \bm{z} \|^2 < 2 h^2 \vspace{2mm} \\
0 & : \| \bm{z} \|^2 \geq 2 h^2
\end{array}
\right.
\end{displaymath}
Hence, denoting the indicator function for the set $\{ \bm{z} : \| \bm{z} \|^2 < 2 h^2 \}$ by
\begin{displaymath}
\mathbbm{1}_{\{ \| \bm{z} \|^2 < 2 h^2 \}} = \left\{
\begin{array}{cr}
1 & : \| \bm{z} \|^2 < 2 h^2 \vspace{2mm} \\
0 & : \| \bm{z} \|^2 \geq 2 h^2
\end{array},
\right.



\begin{displaymath}
\phi(s) = \left\{
\begin{array}{lr}
\phi_1 (s) = (\alpha_j + s)^2 + (\beta_j + s)^2 & : s \in (- \infty, h_j - x_j - \beta_j)\\
\phi_2 (s) = (\alpha_j + s)^2 + (h_j - x_j)^2 & : s \in [h_j - x_j - \beta_j, x_j - \ell_j - \alpha_j]\\
\phi_3 (s) = (x_j - \ell_j)^2 - (h_j - x_j)^2 & : s \in (\ell_j - x_j - \alpha_j, \infty)
\end{array}
\right.
\end{displaymath}
\grid
